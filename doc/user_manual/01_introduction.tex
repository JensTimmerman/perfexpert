\chapter{Introduction}

\section{Purpose}

HPC systems are notorious for operating at a small fraction of their peak performance. The ongoing migration to multi-core and multi-socket compute nodes further complicates performance optimization. The previously available performance optimization tools require considerable effort to learn and use. To enable wide access to performance optimization, TACC and its Technology Insertion partners have developed PerfExpert, a tool that combines a simple user interface with a sophisticated analysis engine to:

\begin{itemize}
  \item Detect and diagnosis the causes for any core-, socket-, and node-level performance bottlenecks in each procedure and loop of an application.
  \item Apply pattern-based software transformations on the application source code to enhance performance on identified bottlenecks.
  \item Provide performance analysis report and suggestions for bottleneck remediation for application's performance bottlenecks which we are unable to optimize automatically.
\end{itemize}

Applying PerfExpert requires only a single command line added to the application's usual job script. PerfExpert automates performance optimization at the core, socket and node levels as far as is possible. PerfExpert works with, and has been tested on, all Intel and AMD processors from Intel Nehalem and AMD 10h microarchitectures. PerfExpert is currently available only for the CPU portion of Stampede compute nodes but will be extended to Intel Many Integrated Cores (MICs) in the near future.

\section{People}

\textbf{James Browne}\\
Professor Emeritus of Computer Science, UT Austin\\
\href{mailto:browne@cs.utexas.edu}{browne@cs.utexas.edu}\\

\noindent\textbf{Leonardo Fialho}\\
Research Scientist, UT Austin\\
\href{mailto:fialho@utexas.edu}{fialho@utexas.edu}\\

\noindent\textbf{Ashay Rane}\\
PhD student, UT Austin\\
\href{mailto:ashay.rane@tacc.utexas.edu}{ashay.rane@tacc.utexas.edu}

\section{Publications}

\begin{itemize}
	\item Ashay Rane, James Browne, \textit{``Enhancing performance optimization of multicore chips and multichip nodes with data structure metrics''}, Parallel Architectures and Compilation Techniques (PACT) 2012.
	\item Ashay Rane, James Browne, Lars Koesterke: \textit{``A Systematic Process for Efficient Execution on Intel's Heterogeneous Computation Nodes''}, Extreme Science and Discovery Environment (XSEDE) 2012.
	\item Ashay Rane, James Browne, Lars Koesterke: \textit{``PerfExpert and MACPO: Which code segments should (not) be ported to MIC?''}, TACC-Intel Highly Parallel Computing Symposium, April 2012.
	\item Ashay Rane, James Browne: \textit{``Performance Optimization of Data Structures Using Memory Access Characterization''}. CLUSTER 2011: 570-574
	\item Ashay Rane, Saurabh Sardeshpande, James Browne: "Determining Code Segments that can Benefit from Execution on GPUs", poster presented at Supercomputing Conference (SC) 2011
	\item M. Burtscher, B.D. Kim, J. Diamond, J. McCalpin, L. Koesterke, and J. Browne. \textit{``PerfExpert: An Easy-to-Use Performance Diagnosis Tool for HPC Applications''}, SC 2010 International Conference for High-Performance Computing, Networking, Storage and Analysis. November 2010
	\item O. A. Sopeju, M. Burtscher, A. Rane, and J. Browne. \textit{``AutoSCOPE: Automatic Suggestions for Code Optimizations Using PerfExpert''}, 2011 International Conference on Parallel and Distributed Processing Techniques and Applications. July 2011
\end{itemize}

\section{Feedback}

If you have problems using PerfExpert on Stampede or Lonestar or suggestions for enhancing PerfExpert, contact us: \href{mailto:fialho@utexas.edu}{\texttt{fialho@utexas.edu}}. If you are reporting a problem, please try to include in your report a compressed file of the \texttt{.perfexpert-temp.XXXXXX} directory generated by the failed execution.

\section{Ways to Contribute}

Version 4 of PerfExpert has been designed to allow third-party contributions. There are several different ways to contribute with PerfExpert, such as:

\begin{itemize}
	\item Providing new bottleneck alleviation solutions.
	\item Creating new strategies to select bottleneck alleviation solutions based on performance metrics.
	\item Adding new performance metrics to PerfExpert.
	\item Writing modules to modify the source code in order to alleviate the identified bottlenecks.
\end{itemize}

Check section \ref{extending} to find some documentation on extending PerfExpert. If you would like to contribute to PerfExpert or need help to do research using PerfExpert, please contact us at \href{mailto:fialho@utexas.edu}{\texttt{fialho@utexas.edu}}. Complete directions on how to add to or modify each phase of PerfExpert can be found on the PerfExpert web site \url{https://bitbucket.org/leonardofialho/perfexpert/}.

\section{Mailing List}

The PerfExpert mailing list is hosted on Google Groups. To subscribe send a message (no content or subject is required) to:
\href{mailto:subscribe-perfexpert@googlegroups.com}{\texttt{subscribe-perfexpert@googlegroups.com}} or access the group's webpage at: \url{https://groups.google.com/d/forum/perfexpert}.

\section{Funding Sources}

The NSF Track 2 Ranger grant and the current NSF Stampede grant.

\section{Acknowledgments}
