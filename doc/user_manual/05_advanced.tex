\chapter{List of Automatic Optimizations PerfExpert Supports}

\begin{description}
	\item[Loop Interchange]\hfill \\
	\item[Loop Tiling (aka Loop Blocking)]\hfill \\
	\item[Loop Fission]\hfill \\

\end{description}

\chapter{PerfExpert for Advanced Users}

This chapter covers details of PerfExpert internal operation. While its reading is not required for common users, advanced user may appreciate its content.

\section{The Big Picture}

\section{Tools Which Are Part of PerfExpert}

This section describes the usage details of each of the tools which compose PerfExpert.

\subsection{\texttt{perfexpert}}

\subsubsection{Synopsis}


\btt perfexpert <\tt\underline{threshold}\btt> [-m \tt\underline{tar}g\underline{et} \btt | -s \tt\underline{sourcefile}\btt] [-r \tt\underline{count} \btt] [-d \tt\underline{database}\btt] [-p \tt p\underline{refix}\btt]\par
\ \ \ \ \ \ \ \ \ \btt[-b \tt\underline{before}\btt] [-a \tt\underline{after}\btt] [-l \tt\underline{level}\btt] \btt[-k \tt\underline{card} \btt[-P \tt p\underline{refix}\btt] [-B \tt\underline{before}\btt] [-A \tt\underline{after}\btt]]\par
\ \ \ \ \ \ \ \ \ \btt[-gchv]\par
\normalfont

\subsubsection{Description}

\subsubsection{Command Line Arguments}


\begin{description}
	\item[\tt threshold]\hfill \\

	\item[\btt -m, --makefile]\hfill \\

	\item[\tt target]\hfill \\

	\item[\btt -s, --source]\hfill \\

	\item[\tt sourcefile]\hfill \\

	\item[\btt -r, --recommendations]\hfill \\
	Set the number of optimizations perfexpert should provide.

	\item[\tt count]\hfill \\
	Number of optimizations perfexpert should provide.

	\item[\btt -d, --database]\hfill \\
	Full path and name of PerfExpert database file (default: \texttt{PERFEXPERT\_VARDIR/RECOMMENDATION\_DB}).

	\item[\tt database]\hfill \\
	Database file (default: \texttt{PERFEXPERT\_VARDIR/RECOMMENDATION\_DB}).

	\item[\btt -p, --prefix]\hfill \\

	\item[\tt prefix]\hfill \\

	\item[\btt -b, --before]\hfill \\

	\item[\btt -a, --after]\hfill \\

	\item[\btt -P, --knc-prefix]\hfill \\

	\item[\btt -B, --knc-before]\hfill \\

	\item[\btt -A, --knc-after]\hfill \\

	\item[\btt -A, --knc-after]\hfill \\

	\item[\btt -g, --clean-garbage]\hfill \\

	\item[\btt -v, --verbose]\hfill \\
	Enable verbose mode (default: level 5).

	\item[\btt -l, --verbose\_level]\hfill \\
	Enable verbose mode using a specific verbose level.

	\item[\tt level]\hfill \\
	Verbose level (range: 1-10).

	\item[\btt -c, --colorful]\hfill \\
	Enable colors on verbose mode, no weird characters will appear on output files.

	\item[\btt -h, --help]\hfill \\
	Show the help message.
\end{description}

\subsubsection{Environment Variables}

Here is a complete list of the environment variables PerfExpert uses. The command line arguments overwrite the value set by any of the environment variables.

\begin{description}
	\item[\texttt{PERFEXPERT\_MAKE\_TARGET}]\hfill \\
	This variable has the same functionality than the \texttt{-m} command line option. It's value is passed to \texttt{make} to compile the source code while using a \texttt{Makefile} (\textit{e.g.}, if it's blue is set to \texttt{all}, PerfExpert will compile the user application using the \texttt{make all} command). If this variable is set while the user selects the \texttt{-s} command line option an error will be generated.

	\item[\texttt{PERFEXPERT\_SOURCE\_FILE}]\hfill \\
	This variable has the same functionality than the \texttt{-s} command line option. It's value should be a valid source code file which will be compiled to obtain the binary executable. If this variable is set while the user selects the \texttt{-m} command line option an error will be generated.

	\item[\texttt{PERFEXPERT\_DATABASE\_FILE}]\hfill \\
	This variable has the same functionality than the \texttt{-d} command line option. It's value should be a valid PerfExpert database file. By default, PerfExpert uses the \texttt{.perfexpert.db} database file located in the user \texttt{\$HOME} directory. If this file does not exist, PerfExpert will copy it from the installation directory.

	\item[\texttt{PERFEXPERT\_REC\_COUNT}]\hfill \\
	This variable has the same functionality than the \texttt{-r} command line option. It's value should be a valid integer number higher than \texttt{0}.

	\item[\texttt{PERFEXPERT\_VERBOSE\_LEVEL}]\hfill \\
	This variable has the same functionality than the \texttt{-l} command line option. It's value should be a valid integer number within \texttt{1} and \texttt{10}.

	\item[\texttt{PERFEXPERT\_COLORFUL}]\hfill \\
	This variable has the same functionality than the \texttt{-c} command line option. It's value should be \texttt{0} or \texttt{1}.
	
	\item[\texttt{CC}]\hfill \\
	This variable sets the compiler PerfExpert should use to obtain the binary executable. It should be a valid executable file. It is possible to set it's value to include the compiler's full path. The value of this variable has no effect when PerfExpert runs without the \texttt{-s} command line option and the \texttt{PERFEXPERT\_SOURCE\_FILE} environment variable is not set. Also, note that the value of this variable has no effect when PerfExpert uses \texttt{Makefile} to obtain the binary executable.

	\item[\texttt{CFLAGS}]\hfill \\
	This variable sets the compiler flags PerfExpert should use while compiling the source code. The value of this variable has no effect when PerfExpert runs without the \texttt{-s} command line option and the \texttt{PERFEXPERT\_SOURCE\_FILE} environment variable is not set. Also, note that the value of this variable has no effect when PerfExpert uses \texttt{Makefile} to obtain the binary executable.

	\item[\texttt{PERFEXPERT\_CFLAGS}]\hfill \\
	This variable is used by PerfExpert to implement optimization which depends upon compiler flags. This variable should not be used to pass any user defined compiler flags because it's content will be freely modified by PerfExpert.

	\item[\texttt{PERFEXPERT\_PREFIX}]\hfill \\
	This variable has the same functionality than the \texttt{-p} command line option. It's value should be a command that will prefix the experiment command line.

	\item[\texttt{PERFEXPERT\_BEFORE}]\hfill \\
	This variable has the same functionality than the \texttt{-b} command line option. It's value should be a command that will be executed before each experiment.

	\item[\texttt{PERFEXPERT\_AFTER}]\hfill \\
	This variable has the same functionality than the \texttt{-a} command line option. It's value should be a command that will be executed after each experiment.

	\item[\texttt{PERFEXPERT\_KNC}]\hfill \\
	This variable has the same functionality than the \texttt{-k} command line option. It's value should be the name of the KNC card (\textit{e.g.}, ``\texttt{mic0}'').

	\item[\texttt{PERFEXPERT\_KNC\_PREFIX}]\hfill \\
	This variable has the same functionality than the \texttt{-P} command line option. It's value should be a command that will prefix the command line that will be executed on the KNC.

	\item[\texttt{PERFEXPERT\_KNC\_BEFORE}]\hfill \\
	This variable has the same functionality than the \texttt{-B} command line option. It's value should be a command that will be executed on the KNC before each experiment.

	\item[\texttt{PERFEXPERT\_KNC\_AFTER}]\hfill \\
	This variable has the same functionality than the \texttt{-A} command line option. It's value should be a command that will be executed on the KNC after each experiment.

\end{description}

\subsubsection{Exit Status}

\subsection{\texttt{analyzer}}

\subsubsection{Synopsis}

\subsubsection{Description}

\subsubsection{Command Line Arguments}

\subsubsection{Exit Status}

\subsection{\texttt{macpo}}

\subsubsection{Synopsis}

\subsubsection{Description}

\subsubsection{Command Line Arguments}

\subsubsection{Exit Status}

\subsection{\texttt{recommender}}

\subsubsection{Description}

\texttt{recommender} is part of PerfExpert. It uses PerfExpert database to select possible optimizations for hot-spots identified by \texttt{analyzer}.

% TODO: describe how  selection of optimization actually works.
     
\subsubsection{Synopsis}

\btt recommender -f \tt\underline{in}p\underline{utfile} \btt[-o \tt\underline{out}p\underline{utfile}\btt] [-a \tt\underline{workdir} \btt] [-r \tt\underline{count}\btt] [-d \tt\underline{database}\btt] [-l \tt\underline{level}\btt]\par
\ \ \ \ \ \ \ \ \ \btt[-m \tt\underline{metricsfile}\btt] [-chnv]\normalfont

\subsubsection{Command Line Arguments}

\begin{description}
	\item[\btt -f, --inputfile]\hfill \\
	Set the file from where performance measurements should be read (default: \texttt{STDIN}).

	\item[\tt inputfile]\hfill \\
	File to use as input for performance measurements. File format: each application hot-spot starts with a percent sign (\texttt{\%}), the \texttt{code} metrics are mandatory, the list of metrics may vary, according to the example above:
	\begin{verbatim}
% 1st application hot-spot
code.section_info=Loop in function compute.omp_fn.3() at ./mm_omp.c:14
code.filename=./mm_omp.c
code.line_number=14
code.type=loop
code.function_name=compute.omp_fn.3
code.extra_info=2
code.representativeness=99.8
code.runtime=22.204
(list of metrics)
...
% 2nd application hot-spot
...
	\end{verbatim}
	\item[\btt -o, --outputfile]\hfill \\
	File to write possible optimizations (default: \texttt{STDOUT}). If the file exists it's content will be overwrite.

	\item[\tt outputfile]\hfill \\
	File to use as output for recommendations (default: \texttt{STDOUT}).

	\item[\btt -a, --automatic]\hfill \\
	Enable automatic performance optimization and create files into a temporary directory (default: off) This option is available only when PerfExpert is compiled with support to ROSE (\url{http://rosecompiler.org/}).

	\item[\tt workdir]\hfill \\
	Temporary directory where intermediary files for automatic performance optimization are created.

	\item[\btt -r, --recommendations]\hfill \\
	Set the number of optimizations perfexpert should provide.

	\item[\tt count]\hfill \\
	Number of optimizations perfexpert should provide.

	\item[\btt -d, --database]\hfill \\
	Full path and name of PerfExpert database file (default: \texttt{PERFEXPERT\_VARDIR/RECOMMENDATION\_DB}).

	\item[\tt database]\hfill \\
	Database file (default: \texttt{PERFEXPERT\_VARDIR/RECOMMENDATION\_DB}).

	\item[\btt -m, --metricsfile]\hfill \\
	Choose a different set of performance metrics. A temporary table will be created, in other words, it automatically enables the \texttt{-n} argument.

	\item[\tt metricsfile]\hfill \\
	File to define metrics different from the default one. File format: each line represents a metric.

	\item[\btt -n, --newmetrics]\hfill \\
	Do not use the system metrics table. A temporary table will be created using the default metrics file (default: \texttt{PERFEXPERT\_ETCDIR/METRICS\_FILE}).

	\item[\btt -v, --verbose]\hfill \\
	Enable verbose mode (default: level 5).

	\item[\btt -l, --verbose\_level]\hfill \\
	Enable verbose mode using a specific verbose level.

	\item[\tt level]\hfill \\
	Verbose level (range: 1-10).

	\item[\btt -c, --colorful]\hfill \\
	Enable colors on verbose mode, no weird characters will appear on output files.

	\item[\btt -h, --help]\hfill \\
	Show the help message.
\end{description}

\subsubsection{Environment Variables}

Here is a complete list of the environment variables \texttt{recommender} uses. The command line arguments overwrite the value set by any of the environment variables. PerfExpert automatically sets most of these variables, but it does not overwrite their values if they are already set. Thus, it is possible to use these variables to pass options to \texttt{recommender}.

\begin{description}
	\item[\texttt{PERFEXPERT\_RECOMMENDER\_INPUT\_FILE}]\hfill \\
	This variable has the same functionality than the \texttt{-f} command line option. It should be a valid readable file. It is possible to set it's value to include a full path. PerfExpert automatically sets the value of this variable.
	
	\item[\texttt{PERFEXPERT\_RECOMMENDER\_OUTPUT\_FILE}]\hfill \\
	This variable has the same functionality than the \texttt{-o} command line option.  It is possible to set it's value to include a full path. PerfExpert automatically sets the value of this variable.

	\item[\texttt{PERFEXPERT\_RECOMMENDER\_DATABASE\_FILE}]\hfill \\
	This variable has the same functionality than the \texttt{-d} command line option. It's value should be a valid PerfExpert database file. By default, PerfExpert sets the value of this variable to the database it is using.

	\item[\texttt{PERFEXPERT\_RECOMMENDER\_METRICS\_FILE}]\hfill \\
	This variable has the same functionality than the \texttt{-m} command line option. It should be a valid readable file. It is possible to set it's value to include a full path.
	
	\item[\texttt{PERFEXPERT\_RECOMMENDER\_REC\_COUNT}]\hfill \\
	This variable has the same functionality than the \texttt{-r} command line option. It's value should be a valid integer number higher than \texttt{0}. By default, PerfExpert sets the value of this variable to value it is set to run with.

	\item[\texttt{PERFEXPERT\_RECOMMENDER\_WORKDIR}]\hfill \\
	The variable controls where \texttt{recommender} will generate intermediary files used for automatic optimization. The value of this variable is automatically set by PerfExpert and it the same working directory the \texttt{perfexpert} itself uses. Usually, it's value is like \texttt{.perfexpert-temp.XXXXXX}.

	\item[\texttt{PERFEXPERT\_RECOMMENDER\_PID}]\hfill \\
	This variable is used by PerfExpert to identify calls to \texttt{recommender} during the same execution but in a different optimization cycle. It is automatically set by PerfExpert and it's value is the PID of the \texttt{perfexpert} process.

	\item[\texttt{PERFEXPERT\_RECOMMENDER\_VERBOSE\_LEVEL}]\hfill \\
	This variable has the same functionality than the \texttt{-l} command line option. It's value should be a valid integer number within \texttt{1} and \texttt{10}. By default, PerfExpert sets the value of this variable to value it is set to run with.

	\item[\texttt{PERFEXPERT\_RECOMMENDER\_COLORFUL}]\hfill \\
	This variable has the same functionality than the \texttt{-c} command line option. It's value should be \texttt{0} or \texttt{1}. By default, PerfExpert sets the value of this variable to value it is set to run with.
\end{description}

\subsubsection{Exit Status}

\texttt{recommender} exits 0 on success, 1 for general errors, and 2 when there is no optimization recommendations available. Any other exit code, even if not specified here, should be considered as an error.

\subsection{\texttt{perfexpert\_ct}}

\subsubsection{Synopsis}

\subsubsection{Description}

\subsubsection{Command Line Arguments}

\subsubsection{Environment Variables}

Here is a complete list of the environment variables \texttt{perfexpert\_ct} uses. The command line arguments overwrite the value set by any of the environment variables. PerfExpert automatically sets most of these variables, but it does not overwrite their values if they are already set. Thus, it is possible to use these environment variables to pass options to \texttt{perfexpert\_ct}.

\begin{description}
	\item[\texttt{PERFEXPERT\_CT\_INPUT\_FILE}]\hfill \\
	This variable has the same functionality than the \texttt{-f} command line option. It should be a valid readable file. It is possible to set it's value to include a full path. PerfExpert automatically sets the value of this variable.

	\item[\texttt{PERFEXPERT\_CT\_OUTPUT\_FILE}]\hfill \\
	This variable has the same functionality than the \texttt{-o} command line option.  It is possible to set it's value to include a full path. PerfExpert automatically sets the value of this variable.

	\item[\texttt{PERFEXPERT\_CT\_DATABASE\_FILE}]\hfill \\
	This variable has the same functionality than the \texttt{-d} command line option. It's value should be a valid PerfExpert database file. By default, PerfExpert sets the value of this variable to the database it is using.

	\item[\texttt{PERFEXPERT\_CT\_WORKDIR}]\hfill \\
	The variable controls where \texttt{recommender} will generate intermediary files used for automatic optimization. The value of this variable is automatically set by PerfExpert and it the same working directory the \texttt{perfexpert} itself uses. Usually, it's value is like \texttt{.perfexpert-temp.XXXXXX}.

	\item[\texttt{PERFEXPERT\_CT\_PID}]\hfill \\
	This variable is used by PerfExpert to identify calls to \texttt{perfexpert\_ct} during the same execution but in a different optimization cycle. It is automatically set by PerfExpert and it's value is the PID of the \texttt{perfexpert} process.

	\item[\texttt{PERFEXPERT\_CT\_VERBOSE\_LEVEL}]\hfill \\
	This variable has the same functionality than the \texttt{-l} command line option. It's value should be a valid integer number within \texttt{1} and \texttt{10}. By default, PerfExpert sets the value of this variable to value it is set to run with.

	\item[\texttt{PERFEXPERT\_CT\_COLORFUL}]\hfill \\
	This variable has the same functionality than the \texttt{-c} command line option. It's value should be \texttt{0} or \texttt{1}. By default, PerfExpert sets the value of this variable to value it is set to run with.
\end{description}

\subsubsection{Exit Status}

\subsubsection{Pattern Recognizers Interface}

\subsubsection{Code Transformers Interface}

\subsection{Other Auxiliary Tools}

\begin{description}
	\item[\texttt{hound}]\hfill \\
	
	\item[\texttt{sniffer}]\hfill \\
\end{description}

\section{PerfExpert Temporary Directory}

Every single time PerfExpert is executed it generates a temporary directory. The name of the directory varies, but it always is in the form of \texttt{.perfexpert-temp.FPAEx6} and is located in the directory from where you called the \texttt{perfexpert} command. Generally, PerfExpert does not remove the temporary directory, unless the user specify in contrary using the \texttt{-g} command line option.

The temporary directory may be useful in many situations, such as:

\begin{itemize}
	\item Re-analyzing results from a previous PerfExpert run.
	\item Checking the output of the binary executable.
	\item Comparing the performance of different version of the same code.
	\item Searching for execution errors.
\end{itemize}

The basic directory tree of the temporary directory looks like the following:

\begin{verbatim}
.perfexpert-temp.FPAEx6
.perfexpert-temp.FPAEx6/1
.perfexpert-temp.FPAEx6/1/database
.perfexpert-temp.FPAEx6/1/database/src
.perfexpert-temp.FPAEx6/1/measurements
.perfexpert-temp.FPAEx6/2
.perfexpert-temp.FPAEx6/2/database
.perfexpert-temp.FPAEx6/2/database/src
.perfexpert-temp.FPAEx6/2/measurements
  (...)
\end{verbatim}

\noindent where for each optimization cycle there is a sub-directory named as cardinal numbers. The sub-directories inside each of the optimization cycle sub-directories are user for:

\begin{description}
	\item[\texttt{database}]\hfill \\
	Stores the \texttt{experiment.xml} file and the sub-directory for the source code. The experiment file, which is generated by \texttt{hpcprof} (a tool from HPCToolkit), is the application performance profile.

	\item[\texttt{database/src}]\hfill \\
	Stores the source code extracted from the binary executable. The full path of the source code will be represented here and also the statically linked libraries.

	\item[\texttt{measurements}]\hfill \\
	Stores the measurement files collected with \texttt{hpcrun} (a tool from HPCToolkit). There is a \texttt{.log} file for each \texttt{hpcrun} invocation and one \texttt{.hpcrun} data file for each thread of each invocation of the binary executable, as following:
\end{description}

\begin{verbatim}
.perfexpert-temp.FPAEx6/1/measurements/mm_omp-000000-000-7281ba5a-91510-0.log
.perfexpert-temp.FPAEx6/1/measurements/mm_omp-000000-000-7281ba5a-91510-0.hpcrun
.perfexpert-temp.FPAEx6/1/measurements/mm_omp-000000-001-7281ba5a-91510-0.hpcrun
  (...)
.perfexpert-temp.FPAEx6/1/measurements/mm_omp-000000-014-7281ba5a-91510-0.hpcrun
.perfexpert-temp.FPAEx6/1/measurements/mm_omp-000000-015-7281ba5a-91510-0.hpcrun
.perfexpert-temp.FPAEx6/1/measurements/mm_omp-000000-000-7281ba5a-91614-0.log
.perfexpert-temp.FPAEx6/1/measurements/mm_omp-000000-000-7281ba5a-91614-0.hpcrun
.perfexpert-temp.FPAEx6/1/measurements/mm_omp-000000-001-7281ba5a-91614-0.hpcrun
  (...)
.perfexpert-temp.FPAEx6/1/measurements/mm_omp-000000-014-7281ba5a-91614-0.hpcrun
.perfexpert-temp.FPAEx6/1/measurements/mm_omp-000000-015-7281ba5a-91614-0.hpcrun
.perfexpert-temp.FPAEx6/1/measurements/mm_omp-000000-000-7281ba5a-91718-0.log
.perfexpert-temp.FPAEx6/1/measurements/mm_omp-000000-000-7281ba5a-91718-0.hpcrun
.perfexpert-temp.FPAEx6/1/measurements/mm_omp-000000-001-7281ba5a-91718-0.hpcrun
  (...)
.perfexpert-temp.FPAEx6/1/measurements/mm_omp-000000-014-7281ba5a-91718-0.hpcrun
.perfexpert-temp.FPAEx6/1/measurements/mm_omp-000000-015-7281ba5a-91718-0.hpcrun
\end{verbatim}

\noindent where the filenames are composed of the name of the binary executable, the MPI JOBID (if existent), the thread ID, the process ID, and other control data. In the example above, we shown the \texttt{.log} files and the \texttt{.hpcrun} files from three invocations of \texttt{hpcrun} (the number of invocations may change from one system to other) using 16 threads each of them.

Besides the sub-directories described above, there are the following files inside each of the optimization cycle sub-directory:

\begin{description}
	\item[\texttt{recommender\_report.txt}]\hfill \\

	\item[\texttt{analyzer\_report.txt}]\hfill \\

	\item[\texttt{analyzer\_metrics.txt}]\hfill \\

	\item[\texttt{mm\_omp.hpcstruct}]\hfill \\

	\item[\texttt{hpcstruct.output}, \texttt{hpcrun.1.output}, \texttt{hpcrun.2.output}, \texttt{hpcrun.3.output}, \texttt{hpcprof.output}]\hfill \\

\end{description}

\section{Analyzing the Results from a Previous PerfExpert Run}

\section{Executing on the Intel\textsuperscript{\textregistered} Xeon Phi\textsuperscript{\texttrademark} Coprocessors}

\chapter{Extending PerfExpert}
\label{extending}

\section{PerfExpert Database Layout}

\section{Adding Metrics to PerfExpert}

\subsection{Performance Counters}

\subsection{Derived Metrics}

\section{New Recommendations for Optimization}

\section{Enabling New Automatic Optimizations}

\subsection{Pattern Recognizers Interface}

\subsection{Code Transformers Interface}
