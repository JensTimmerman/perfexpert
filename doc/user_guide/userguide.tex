\documentclass[titlepage]{report}

\usepackage{hyperref}
\usepackage{tikz}
\usetikzlibrary{shapes,shadows}
\tikzstyle{abstractbox}=[draw=black,fill=white,rectangle,inner sep=10pt,style=rounded corners,drop shadow={fill=black,opacity=1}]
\tikzstyle{abstracttitle}=[fill=white]
 
\newcommand{\nicebox}[3][fill=white]{
	\begin{center}
		\begin{tikzpicture}
			\node [abstractbox, #1] (box)
			{\begin{minipage}{0.95\linewidth}
				#2
			\end{minipage}};
			\node[abstracttitle, right=10pt] at (box.north west) {\textbf{#3}};
		\end{tikzpicture}
	\end{center}
}

\begin{document}

\title{{\Huge PerfExpert v4.0}\\ \ \\
	{\huge User Guide}\\ \ \\
	An Easy-to-Use Automatic Performance Diagnosis and Optimization Tool\\for HPC Applications}
\author{\\ \ \\ Leonardo Fialho\\
	\href{mailto:fialho@utexas.edu}{fialho@utexas.edu}\\ \ \\
	Ashay Rane\\
	\href{mailto:ashay.rane@tacc.utexas.edu}{ashay.rane@tacc.utexas.edu}\\ \ \\
	James Browne\\
	\href{mailto:browne@cs.utexas.edu}{browne@cs.utexas.edu}\\ \ \\
	\href{http://www.tacc.utexas.edu}{Texas Advanced Computing Center}\\
	\href{http://www.utexas.edu}{The University of Texas at Austin}}
\date{August 2013\\
	Revision 3}
\maketitle

\tableofcontents

\chapter{Introduction}

\section{Purpose}

HPC systems are notorious for operating at a small fraction of their peak performance, and the ongoing migration to multi-core and multi-socket compute nodes further complicates performance optimization.

The previously available performance optimization tools require considerable effort to learn and use. To enable wide access to performance optimization, TACC and its technology insertion partners have developed PerfExpert, a tool that combines a simple user interface with a sophisticated analysis engine to:

\begin{itemize}
  \item Detect and diagnosis the causes for any core, socket, and node-level performance bottlenecks in each procedure and loop of an application;
  \item Apply pattern-based software transformations on the application source code to enhance performance on identified bottlenecks;
  \item Provide performance analysis report and suggestions for bottleneck remediation for application's performance bottlenecks which we are unable to optimize automatically.
\end{itemize}

Applying PerfExpert requires only a single command line added to the applications normal job script. PerfExpert automates performance optimization at the core, socket and node levels as far as is possible. PerfExpert is currently available only for the CPU portion of Stampede compute nodes but will be extended to Intel MICs in the near future.

\section{Funding Sources}

The NSF Track 2 Ranger grant and the current NSF Stampede grant.

\section{TACC Staff}

\textbf{James Browne}\\
Professor Emeritus of Computer Science, UT Austin\\
\href{mailto:browne@cs.utexas.edu}{browne@cs.utexas.edu}\\

\noindent\textbf{Leonardo Fialho}\\
Research Scientist, UT Austin\\
\href{mailto:fialho@utexas.edu}{fialho@utexas.edu}\\

\noindent\textbf{Ashay Rane}\\
PhD candidate, UT Austin\\
\href{mailto:ashay.rane@tacc.utexas.edu}{ashay.rane@tacc.utexas.edu}

\section{Publications}

\begin{itemize}
	\item Ashay Rane, James Browne, Lars Koesterke: \textit{``PerfExpert and MACPO: Which code segments should (not) be ported to MIC?''}, TACC-Intel Highly Parallel Computing Symposium, April 2012.
	\item Ashay Rane, James Browne: \textit{``Performance Optimization of Data Structures Using Memory Access Characterization''}. CLUSTER 2011: 570-574
	\item Ashay Rane, Saurabh Sardeshpande, James Browne: "Determining Code Segments that can Benefit from Execution on GPUs", poster presented at Supercomputing Conference (SC) 2011
	\item M. Burtscher, B.D. Kim, J. Diamond, J. McCalpin, L. Koesterke, and J. Browne. \textit{``PerfExpert: An Easy-to-Use Performance Diagnosis Tool for HPC Applications''}, SC 2010 International Conference for High-Performance Computing, Networking, Storage and Analysis. November 2010
	\item O. A. Sopeju, M. Burtscher, A. Rane, and J. Browne. \textit{``AutoSCOPE: Automatic Suggestions for Code Optimizations Using PerfExpert''}, 2011 International Conference on Parallel and Distributed Processing Techniques and Applications. July 2011
\end{itemize}

\section{Feedback}

If you have problems using PerfExpert on Stampede or Lonestar or suggestions for enhancing PerfExpert, contact us: \href{mailto:fialho@utexas.edu}{\texttt{fialho@utexas.edu}}. If you are reporting a problem, please try to include in your report a compressed file of the \texttt{/.perfexpert-temp.XXXXXX} directory generated by the failed execution.

\section{Want to Contribute?}

Version 4 of PerfExpert has been designed to allow third-party contributions. There are several different ways to contribute with PerfExpert, such as:

\begin{itemize}
	\item Providing new bottleneck alleviation solutions;
	\item Creating new strategies to select bottleneck alleviation solutions based on performance metrics;
	\item Adding new performance metrics to PerfExpert;
	\item Writing modules to modify the source code in order to alleviate the identified bottlenecks.
\end{itemize}

If you want to contribute with PerfExpert or need help to do research using PerfExpert, contact us: \href{mailto:fialho@utexas.edu}{\texttt{fialho@utexas.edu}}. Full directions on how to add to or modify each phase of PerfExpert can be found on the PerfExpert web site \href{https://bitbucket.org/leonardofialho/perfexpert/}{\texttt{https://bitbucket.org/leonardofialho/perfexpert/}}. Please tell us how we can we help you to help us.

\section{Mailing List}

[WE HAVE TO CREATE A MAILING LIST]

\chapter{Installation Instructions}

\section{Prerequisites}

PerfExpert is based on other tools so that installation of PerfExpert requires that they be installed. These tools are:

\begin{itemize}
	\item \textbf{PAPI} (\href{http://icl.cs.utk.edu/papi/software/}{\texttt{http://icl.cs.utk.edu/papi/software/}}): PAPI is required to measure hardware performance metrics like cache misses, branch instructions, etc. The PAPI installation is mostly straightforward: download, \texttt{./configure}, \texttt{make}, and \texttt{make install}. If your Linux kernel version is 2.6.32 or higher, then PAPI will mostly likely use \texttt{perf\_events}. Recent versions of PAPI (v3.7.2 and beyond) support using \texttt{perf\_events} in the Linux kernel. However, if your kernel version is lower than 2.6.32, then you would require patching the kernel with either \texttt{perfctr}\footnote{\href{http://user.it.uu.se/~mikpe/linux/perfctr/2.6/}{\texttt{http://user.it.uu.se/$\sim$mikpe/linux/perfctr/2.6/}}} or \texttt{perfmon}\footnote{\href{http://perfmon2.sourceforge.net/}{\texttt{http://perfmon2.sourceforge.net/}}}.
	\item \textbf{HPCToolkit} (\href{http://hpctoolkit.org/software.html}{\texttt{http://hpctoolkit.org/software.html}}): HPCToolkit is a tool that works on top of PAPI. HPCToolkit is used by PerfExpert to run the program multiple times with specific performance counters enabled. It is also useful for correlating addresses in the compiled binary back to the source code.
	\item \textbf{Java Virtual Machine} and the \textbf{Java Development Kit}\\(\href{http://www.oracle.com/technetwork/java/javase/downloads/}{\texttt{http://www.oracle.com/technetwork/java/javase/downloads/}}).
	\item \textbf{ROSE Compiler} (\href{http://rosecompiler.org/}{\texttt{http://rosecompiler.org/}}): ROSE is the framework PerfExpert uses to manipulate the application�s source code. It is required only if you wish to compile PerfExpert with performance optimization capability.
	\item \textbf{Apache Ant} (\href{http://ant.apache.org/bindownload.cgi}{\texttt{http://ant.apache.org/bindownload.cgi}}): it is required only to compile PerfExpert
	\item \textbf{SQLite} version 3 (\href{http://www.sqlite.org/}{\texttt{http://www.sqlite.org/}}): PerfExpert uses a SQLite database to store suggestions for bottleneck remediation and other information for automatic optimization.
	\item \textbf{GNU Multiple Precision Arithmetic Library} (\href{http://gmplib.org/}{\texttt{http://gmplib.org/}}): MACPO, one of the tools which compose PerfExpert requires this package.
	\item \textbf{Google SparseHash} (\href{https://code.google.com/p/sparsehash/}{\texttt{https://code.google.com/p/sparsehash/}}):\\MACPO, one of the tools which compose PerfExpert requires this package.
\end{itemize}

\section{Downloading PerfExpert}

PerfExpert is an open-source project. Funding to keep researchers working on PerfExpert depends on the value of this tool to the scientific community. For that reason, it is really important to know where and who are using our tool. Please, send us a message (\href{mailto:fialho@utexas.edu}{\texttt{fialho@utexas.edu}}) telling what is the institution (name and country) you are planning to install and test PerfExpert and fell free to explain (or not) why you want to try PerfExpert. Many of the collaboration we have today started with simple contacts like that. Keep in mind you will also be helping us to maintain PerfExpert.

The PerfExpert source code can be downloaded from the registration page: \href{http://www.tacc.utexas.edu/perfexpert/registration}{\texttt{http://www.tacc.utexas.edu/perfexpert/registration}}.

We encourage people to have a look on the wiki page, send patches, and raise issues on PerfExpert�s page on BitBucket:\\\href{https://bitbucket.org/leonardofialho/perfexpert/}{\texttt{https://bitbucket.org/leonardofialho/perfexpert/}}.

\section{Setting up PerfExpert}

PerfExpert now comes with a \texttt{Makefile}-base source code tree to automate the entire installation process. Thus the compilation and installation of PerfExpert is similar to any other GNU package:

\begin{verbatim}
$ ./configure
$ make
$ make install
\end{verbatim}

Optionally, you may want to run the set of test we have included into PerfExpert. To do so, just run ``\texttt{make check}'' after compiling your code.

If any of the prerequisites of PerfExpert is not on the \texttt{PATH} or \texttt{LD\_LIBRARY\_PATH}, you can specify the right locations of such files. For that, have a look on the configure script help.

\begin{verbatim}
$ ./configure --help
\end{verbatim}

If you have downloaded PerfExpert from our control version system, probably you have already noticed that there is no configure script available on the source code tree. To generate it run the following command:

\begin{verbatim}
$ ./autogen.sh
\end{verbatim}

\texttt{autogen.sh} requires the following packages available:

\begin{itemize}
	\item M4 version 1.4.13 or newer (\href{ftp://ftp.gnu.org/gnu/m4/}{\texttt{ftp://ftp.gnu.org/gnu/m4/}})
	\item Autoconf version 2.63 or newer (\href{ftp://ftp.gnu.org/gnu/autoconf/}{\texttt{ftp://ftp.gnu.org/gnu/autoconf/}})
	\item Automake version 1.11.1 or newer (\href{ftp://ftp.gnu.org/gnu/automake/}{\texttt{ftp://ftp.gnu.org/gnu/automake/}})
	\item Libtool version 2.2.6b or newer (\href{ftp://ftp.gnu.org/gnu/libtool/}{\texttt{ftp://ftp.gnu.org/gnu/libtool/}})
\end{itemize}

In case you have any problem installing PerfExpert, send us an email\\ (\href{fialho@utexas.edu}{\texttt{fialho@utexas.edu}}) or use our mailing list: [WE HAVE TO CREATE A MAILING LIST].

\section{Characterizing your Machine}

During the installation process, PerfExpert will run a set o benchmark applications to characterize your machine. This characterization is used to analyze the performance of the applications you want to optimize. For that reason, you should be sure the PerfExpert installation process is done on a machine of the same kind you are planning to run the production code. If it is not possible, you should run both the benchmark application and generate the characterization file manually and move it to the etc directory where PerfExpert has been installed.

\chapter{Using PerfExpert}

The objective of this chapter is to explain how to run programs using PerfExpert and how to interpret its output using a simple matrix multiplication program. In this chapter, we will use the OpenMP simple matrix multiplication program\footnote{\href{https://computing.llnl.gov/tutorials/openMP/samples/C/omp\_mm.c}{\texttt{https://computing.llnl.gov/tutorials/openMP/samples/C/omp\_mm.c}}}. This program multiplies two matrices and prints one value from the resulting matrix.

\nicebox{PerfExpert may, if you choose to use the full capabilities for automated optimization, change your source code during the process of optimization. PerfExpert  always saves the original file with a different name (\textit{e.g.}, \texttt{omp\_mm.c.old\_27301}) as well as adding annotations to your source code for each optimization it makes. We cannot, however, fully guarantee that code modifications for optimizations will not break your code. We recommend having a full backup of your original source code before using PerfExpert.}{CAUTION:}

\section{Environment Configuration}

If you are using any of the TACC\footnote{\href{https://www.tacc.utexas.edu/}{\texttt{https://www.tacc.utexas.edu/}}} machines, load the appropriate modules:

module load papi perfexpert

The runs with PerfExpert should be made using a data set size for each compute node which is equivalent to full production runs but for which execution time is not more than about ten or fifteen minutes since PerfExpert will run your application multiple times (actually, three times on Stampede) with different performance counters enabled. For that reason, before you run PerfExpert you should either request iterative access to computational resources (compute node), or modify the job script that you use to run your application and specify a running time that is about 3 (for Stampede) or 6 (for Lonestar) times the normal running time of the program.

To request iterative access to a compute node on Stampede, please, have a look on the User Guide\footnote{\href{https://portal.tacc.utexas.edu/group/tup/user-guides/stampede\#running}{\texttt{https://portal.tacc.utexas.edu/group/tup/user-guides/stampede\#running}}}.

Below is an example of a job script file modified to use PerfExpert which runs PerfExpert on the application named my\_program and generate the performance analysis report. Adding command line options will cause suggestions for bottleneck remediation to be generated and output and/or automatic performance optimization to be attempted. 

\noindent\hrulefill
\begin{verbatim}
#!/bin/bash

# job name
#SBATCH -J myMPI

# output and error filename (%j stands to jobID)
#SBATCH -o myMPI.o%j

# total number of mpi tasks requested
#SBATCH -n 16

# queue (partition) -- normal, development, etc.
#SBATCH -p development

# run time (hh:mm:ss) - 1.5 hours
#SBATCH -t 01:30:00

# run the executable named my_program
perfexpert 0.1 ./my_program
\end{verbatim}\hrulefill
	
\section{PerfExpert Options}

There are several different options for applying PerfExpert. The following summary shows you how to choose the options to run PerfExpert to match your needs.

\begin{verbatim}
$ perfexpert -h
Usage: perfexpert <threshold> [-gvch] [-l level] [-d database]
                  [-r count] [-m target|-s sourcefile] [-p prefix]
                  [-a FILE] [-b FILE] <program_executable>
                  [program_arguments]

 <threshold>        Define the relevance (in % of runtime) of
                    code fragments PerfExpert should take into
                    consideration (> 0 and <= 1)
 -d --database      Select the recommendation database file
 -r --recommend     Number of recommendation to show
 -m --makefile      Use GNU standard 'make' command to compile
                    the code (it requires the source code
                    available in current directory)
 -s --source        Specify the source code file (if your source
                    code has more than one file please use a
                    Makefile and choose -m option it also enables
                    the automatic optimization (-a)
 -p --prefix        Add a prefix to the command line (e.g.
                    mpirun) use double quotes to specify multiple
                    arguments (e.g. -p "mpirun -n 2"
 -b --before        Execute FILE before each run of the
                    application
 -a --after         Execute FILE after each run of the
                    application
 -v --verbose       Enable verbose mode using default verbose
                    level (1)
 -l --verbose_level Enable verbose mode using specific verbose
                    level (1-10)
 -c --colorful      Enable colors on verbose mode, no weird
                    characters will appear on output files
 -h --help          Show this message

Use CC, CFLAGS and LDFLAGS to select compiler and compilation/
linkage flags
\end{verbatim}

If you select the \texttt{-�m} or \texttt{-�s} options, PerfExpert will try to automatically optimize your code and shows you the performance analysis report and the list of suggestion for bottleneck remediation when no automatic optimization is possible.

For the \texttt{-�m} or \texttt{-�s} options, PerfExpert requires access to the application source code. If you select the \texttt{�-m} option and the application is composed of multiple files, your source code tree should have a \texttt{Makefile} file to enable PerfExpert compile your code. If your application is composed of a single source code file, the option \texttt{-�s} is sufficient for you. If you do not select \texttt{-�m} or \texttt{-�s} options, PerfExpert requires only the binary code and will show you only the performance analysis report and the list of suggestion for bottleneck remediation.

PerfExpert will run your application multiple times to collect different performance metrics. You may use the \texttt{-�b} or \texttt{-�a} options if you want to execute a program or script between each run. The argument \texttt{program\_executable} should be the filename of the application you want to analyze, not a shell script, otherwise, PerfExpert will analyze the performance of the shell script instead of the performance of you application.

Use the \texttt{-r} option to select the number of recommendations for optimization you want for each code section which is a performance bottleneck.

\nicebox{If your program takes any argument that starts with a ``--�'' signal PerfExpert will interpret this as a command line option. To help PerfExpert handle \texttt{program\_arguments} correctly, use quotes and add a space before the program�s arguments (e.g., \texttt{" -�s 50"}).}{CAUTION:}

\nicebox{In the case you are trying to optimize a MPI application, you should use the \texttt{-�p} option to specify the MPI launcher and also it's arguments.}{CAUTION:}

For this guide, using the OpenMP simple matrix multiply code, we will use the following command line options:

\begin{verbatim}
$ OMP_NUM_THREADS=16 CFLAGS="-fopenmp" perfexpert -�s mm_omp.c 0.05
  mm_omp
\end{verbatim}

which executes PerfExpert's automatic optimizations and will also generate an OpenMP-enabled binary which will run with 16 threads. In this case, PerfExpert will compile the \texttt{mm\_omp.c} code using the systems' default compiler, which is GCC in the case of Stampede. PerfExpert will take into consideration only code fragments (loops and functions) that take more 5\% of the runtime.

To select a different compiler, you should specify the \texttt{CC} environment variable as below:

\begin{verbatim}
$ CC="icc" OMP_NUM_THREADS=16 CFLAGS="-fopenmp" perfexpert �-s
  mm_omp.c 0.05 mm_omp
\end{verbatim}

\section{The Performance Analysis Report}

This section explains the performance analysis report and the metrics shown by PerfExpert. We discuss the following sample output:

\begin{verbatim}
Loop in function compute() (99.9% of the total runtime)
=========================================================================
ratio to total instrns       %  0.......25.........50.......75......100
   - floating point      :    6 ***
   - data accesses       :   33 ****************

* GFLOPS (% max)         :    7 ***
   - packed              :    0 *
   - scalar              :    7 ***
-----------------------------------------------------------------------
performance assessment     LCPI good....okay....fair....poor....bad...
* overall                :  0.8 >>>>>>>>>>>>>>>>
upper bound estimates
* data accesses          :  2.4 >>>>>>>>>>>>>>>>>>>>>>>>>>>>>>>>>>>>>>+
   - L1d hits            :  1.3 >>>>>>>>>>>>>>>>>>>>>>>>>
   - L2d hits            :  0.3 >>>>>
   - L2d misses          :  0.8 >>>>>>>>>>>>>>>>
* instruction accesses   :  0.3 >>>>>>
   - L1i hits            :  0.3 >>>>>>
   - L2i hits            :  0.0 >
   - L2i misses          :  0.0 >
* data TLB               :  1.5 >>>>>>>>>>>>>>>>>>>>>>>>>>>>>>>
* instruction TLB        :  0.0 >
* branch instructions    :  0.0 >
   - correctly predicted :  0.0 >
   - mispredicted        :  0.0 >
* floating-point instr   :  0.2 >>>>
   - fast FP instr       :  0.2 >>>>
   - slow FP instr       :  0.0 >
\end{verbatim}

Apart from the total running time, PerfExpert performance analysis report includes, for each code segment:

\begin{itemize}
	\item Instruction execution ratios (with respect to total instructions);
	\item Approximate information about the computational efficiency (GFLOPs measurements);
	\item Overall performance;
	\item Local Cycles Per Instruction (LCPI) values for the cost of memory accesses.
\end{itemize}

The program composition part shows what percentage of the total instructions were computational (floating-point instructions) and what percentage were instructions that accessed data. This gives a rough estimate in trying to understand whether optimizing the program for either data accesses or floating-point instructions would have a significant impact on the total running time of the program.

The PerfExpert performance analysis report also shows the GFLOPs rating, which is the number of floating-point operations executed per second in multiples of 109. The value for this metric is displayed as a percentage of the maximum possible GFLOP value for that particular machine. Although it is rare for real-world programs to match even 50\% of the maximum value, this metric can serve as an estimate of how efficiently the code performs computations.

The next, and major, section of the PerfExpert performance analysis report shows the LCPI values, which is the ratio of cycles spent in the code segment for a specific category, divided by the total number of instructions in the code segment. The overall value is the ratio of the total cycles taken by the code segment to the total instructions executed in the code segment. 

Generally, a value of 0.5 or lower for an LCPI is considered to be good. However, it is only necessary to look at the ratings (\texttt{good}, \texttt{okay}, \ldots, \texttt{bad}) The rest of the report maps this overall LCPI, into the six constituent categories: data accesses, instruction accesses, data TLB accesses, instruction TLB accesses, branches and floating point computations. Without getting into the details of instruction operation on Intel and AMD chips, one can say that these six categories record performance in non-overlapping ways. That is, they roughly represent six separate categories of performance for any application.

The LCPI value is a good indicator of the cost arising from instructions of the specific category. Hence, the higher the LCPI, the slower the program. The following is a brief description of each of these categories:

\begin{description}
	\item[Data accesses:]\hfill \\
	counts the LCPI arising from accesses to memory for program variables.
	\item[Instruction accesses:]\hfill \\
	counts the LCPI arising from memory accesses for code (functions and loops).
	\item[Data TLB:]\hfill \\
	provides an approximate measure of penalty arising from strides in accesses or regularity of accesses.
	\item[Instruction TLB:]\hfill \\
	reflects cost of fetching instructions due to irregular accesses.
	\item[Branch instructions:]\hfill \\
	counts cost of jumps (i.e. if statements, loop conditions, etc.).
	\item[Floating-point instructions:]\hfill \\
	counts LCPI from executing computational (floating-point) instructions.
\end{description}

Some of these LCPI categories have subcategories. For instance, the LCPI from data and instruction accesses can be divided into LCPI arising from the individual levels of the data and instruction caches and branch LCPIs can be divided into LCPIs from correctly predicted and from mispredicted branch instructions. For floating-point instructions, the division is based on floating-point instructions that take few cycles to execute (\textit{e.g.}, add, subtract and multiply instructions) and on floating-point instructions that take longer to execute (\textit{e.g.}, divide and square-root instructions).

In each case, the classification (data access, instruction access, data TLB, etc.) is shown so that it is easy to understand which category is responsible for the performance slowdown. For instance if the overall CPI is ``\texttt{poor}'' and the data access LCPI is high, then you should concentrate on access to program variables and memory. Additional LCPI details help in relating performance numbers to the process architecture.

\nicebox{When PerfExpert runs with automatic performance optimization enabled the performance analysis report shown reflects the performance of the code after all possible automatic optimizations have been applied.}{IMPORTANT:}

\nicebox{PerfExpert creates a \texttt{/.perfexpert-temp.XXXXXX} directory for each time it is executed. This directory has one subdirectory for each optimization cycle PerfExpert completed or attempted. Each subdirectory includes the intermediate files PerfExpert generated during each cycle, including the performance analysis reports.}{NOTICE:}

\section{List of Recommendations for Optimization}

If PerfExpert runs with \texttt{-�r} option enabled, it will generate the performance analysis report  and a list of suggestions for bottleneck remediation for each bottleneck. This option is always available, it does not depend on which of the other command line option are.

A list of suggestions for this example run (truncated to only the most important recommendation) is shown following. Each entry in this list is similar to the following one:

\begin{verbatim}
Loop in function compute() at mm_omp.c:15 (100% of the total runtime) 
*********************************************************************
change the order of loops
this optimization may improve the memory access pattern and make it
more cache and TLB friendly.
loop i {
  loop j {...}
}
=====>
loop j {
  loop i {...}
}
\end{verbatim}

\end{document}
